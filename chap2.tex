\def\BM{Brownian Motion}
\chapter{Properties of \BM}

\BM s have a lot of the general properties an SP might have.
Hence, we can do a lot with \BM s we can not do in a more general setting.

\section{Invariance properties}
The statements are far more interesting than the proofs,
which are too easy to help in understanding the properties.
We can do all kinds of funny things to \BM and obtain \BM s again.

When we just transform the values of a \BM,
we can get new \BM s quite easily.

\begin{prop}[Orthogonal invariance]
	Let \(\B\) be a \def\BMd{\BM\textsuperscript{d}}\BMd
	and \(U\) d by d orthogonal matrix
	then \((U\B_t)_{t\in T}\) is a \BMd.
	In particular, \(-\B = (-\B_t)_{t\in T}\) is a \BMd.
\end{prop}
% Kasten

\begin{proof}
	With \(UU^* = \Id\) we have …
\end{proof}

Instead of transforming the values,
we can also shift the process in time.
This is an important property which to understand will be quite usefull.

\begin{prop}[Time shift invariance]
\end{prop}
% Kasten
% TODO: Graphic depiction of time shift (lots o blood)
%\begin{tikzpicture}
%	\newcommand*\coordinates[2][]{%
%		\begin{scope}}
%\end{tikzpicture}

The next property is very fundamental
and central to our understanding of \BM s.
The path a \BM takes is entirely independent of the path taken so far
except for the value the \BM now starts at.
If we forget where we are
we essentially delete all memory of the past.
For all relevant purposes,
the value we take at the current point in time
is all we know about what has happened so far.

\begin{prop}[Memoryless property, elementary Markov property]
	\(\timeshift\B a\)
\end{prop}
% Kasten

\begin{proof}
	Idea: fdd and intersection stable generator of the \(\sigma\)-algebras
	Transform the fdd with \[A = \lowtriangones\]
	to get another int stable generator.
	Do this for both processes.
	\[W_{t_i}-W_{t_{i-1}} = B_{t_i+a}-B_{t_{i-1}+a}\]
	Use (B1).
	P-Theo: Indep of int-stable generator means indep of sigma-alg.
\end{proof}

A diffusive rescaling is a rescaling of a process
where we rescale the time,
\eg let it run twice as fast,
where we also transform the values to offset the rescaling.
If we were to double the speed of a linear function
and then were to divide the values by \(2\)
we would obtain the original function again.
In case of \BM we obviously will not be able to recover the exact function
but we can get a process of identical distribution.

\begin{prop}[Invariance under diffusive rescaling]
\end{prop}
% Kasten

\begin{proof} Exercise\end{proof}

Imagine ink in water.
If we let time run 100 times as fast
the ink will seem to disperse at (merely) ten times the speed.

The function recovered exactly by this rescaling is the square root.
This means that a \BM looks “locally like a square root except it doesn't”.
\BM very rough and does not look like sqrt at all.
(It only looks like sqrt on small and large scales. –.–)
Sqrt leaves \(0\) very fast but gets slower over time.

If we stop a \BM at some point in time
and let it play backwards
we obtain a \BM again.

\begin{prop}[Time reversal symmetry]
\end{prop}
% Kasten
\begin{proof} Exercise\end{proof}

Another way in which \BM s look like the sqrt:
\[\sqrt t = t\sqrt{\frac1t}\]

relates behaviour at zero to behaviour at infinity
How a bm looks at infinity how it looks at zero
a very strong symmetry

The involution is not a \BM on the original probability space
since there will be paths that are not continuous in zero.
However, nearly all paths are continuous
and we can simply throw away the remaining points.

\begin{prop}[Time involution invariance]
\end{prop}
% Kasten
\begin{proof}
	The hard part of the proof is continuity at zero

	Something that is often very useful when dealing with SP:
	Write out definition of the limit in quantors over countable sets.
	Intersection with \(\Q\) is sufficient
	since we have continuity outside of zero % FIXME Really? Needs details

	\(\mathcal F \cap \Omega_0 \coloneqq \{F \cap \Omega_0 : F \in \mathcal F\}\)
\end{proof}

We now change our perspective.
So far, we took values and time and did things to that.
We took a function from R to R and transformed it in some way.
All of these can be written as operators on function spaces
to obtain measure-preserving maps.

\begin{bem}[Invariances of Wiener measure]
\end{bem}

Think of \(C_0\) like a big \(\R^n\).
Many maps will not preserve the measure but some do.

\section{Martingale properties of \BM}

In probability theory often only martingales on discrete sets are defined
but it works just as well on continuous sets.

\begin{defi}
	The third condition is the one that makes a martingale a martingale

	sub/super harmonic functions

	\def\bup{\sffamily b\hspace*{-7.6pt}\raisebox{3pt}{\(\uparrow\)}}
	\def\pdown{\sffamily p\hspace*{-7.6pt}\raisebox{-3pt}{\(\downarrow\)}}
	\def\submart{\textsf{su\rlap{\bup}\phantom{b}-martingale}}
	\def\supmart{\textsf{su\rlap{\pdown}\phantom{p}-martingale}}
	\submart\ \supmart
\end{defi}
% Kasten

%Here seventh lecture starts, only notes, nothing complete
\begin{prop}[2.10]
BM is an $\mathcal{F}_t^B$ martingale
\end{prop}

Later we will see that $\mathcal{F}^B$ is sometimes to small.
Replacing it by a larger filtration can in principle destroy the martingale property.

\begin{defi}[admissble filtration]

\end{defi}

Now, we check that a BM is a martingale wrt an admissible martingale.
\begin{prop}
BM is an $\mathcal{F}_t$ martingale.
\end{prop}
\begin{proof}
same as for 2.10 as we used only independence of increments.
\end{proof}

\begin{prop}
$(B_t)$ BM $\mathcal{F}_t$ admissible.
Then the following SPs are martingales.
\begin{enumerate}[label=\roman*)]
\item $M_t\coloneqq \abs{B_t}^2-dt$
\item $M_t^v \coloneqq \mathrm{e}^{(v,B_t)-\frac{t}{2}\abs{v_R}^2+\frac{t}{2}\abs{v_I}^2}$ for all $v=v_R+\i V_I \in \C^d$. (for real $\xi$ this may look like a Fourier transform).
%TODO immernoch falsch
\item For $d=1$, $M_t^n \coloneqq t^{\frac{n}{2}} H_n(t^{-\frac{1}{2}}B_t)$ where $H_n$ is the $n$-th Hermite polynomial, \ie ,
$H_n(x)=(-1)^{\frac{1}{2}} \mathrm{e}^{-\frac{x^2}{2}} \partial_x^n(\mathrm{e}^{-\frac{x^2}{2}})$.
These Hermite polynomials are an orthogonal basis in a weighted $\mathit{L}^2$ space, \ie ,
\begin{align*}
\int H_n(x)H_m(x) \mathrm{e}^{-\frac{x^2}{2}}~\mathrm{d}x=\delta_{n,m}
\end{align*}
when weighted by the correct factor (probably some $\frac{1}{\sqrt{2\pi}})$.
\end{enumerate}
\end{prop}

\begin{proof}
\begin{enumerate}[label=\roman*)]
\item It suffices to consider the case $d=1$ since
\begin{align*}
\abs{B_t}^2-dt=\sum_{j=1}^d \left(\abs{B_t^j}^2-t\right).
\end{align*}
We need to introduce increments in order to use the properties on an admissible filtration.
\begin{align*}
\E{B_t^2\mid \mathcal{F}_s}=\E{(B_t-B_2)^2 +2 B_s B_t-B_s^2 \mid \F_s}
\end{align*}
and use linearity and independence.
\item First need to check integrability, $\E{\abs{M_t^v}\infty}$ by Corollary 1.12
%TODO number correct?
Now, $\E{M_t^v\mid \F_s}=\mathrm{e}^{-\frac{t}{2}\abs{v}^2} \mathrm{e}^{(v,B_2)} \E{\mathrm{e}^{(v,B_t-B_s)}\mid \F_s}$
and use what we calculated in 1.12.
\item exercise: expand $M_t^\alpha\coloneqq \e^{\alpha B_t}\e^{-\alpha^2 \frac{t}{2}}$
and sort by powers of $\alpha$.
\end{enumerate}
\end{proof}

All martingales so far were of the form $M_t=f(t,B_t)$ for some function $f$.
But there are more general martingales.
In order to construct those we need following important lemma.
It is a classical theorem in theory of PDEs.

\begin{lem}
The transition density
\begin{align*}
p_t(x)=\frac{1}{(2\pi t)^\frac{d}{2}} \exp(-\frac{\abs{x}^2}{2t})
\end{align*}
of BM$^d$ solves the \emph{(scaled) heat equation}, \ie ,
\begin{align*}
\partial_t p_t(x)=\frac{1}{2}\Delta p_t(x)
\end{align*}
for all $t>0$ and $x \in \Rd$.
\end{lem}
\begin{proof}
We consider the Fourier transform of the density as then it will be easier to calculate derivatives
\begin{align*}
\hat{p}_t(k)\coloneqq \int \e^{\i(k,x)}p_t(x)~\mathrm{d}x=\e^{-\frac{1}{2}\abs{k}^2t}
\end{align*}
Clearly $\partial_t \hat{p}_t(k)=-\frac{1}{2}\abs{k}^2\hat{p}_t(k)$ and by Dominated convergence we may change derivative and integration
\begin{align*}
\int \e^{\i(k,x)}\partial_t p_t(x)~\mathrm{d}x=-\frac{1}{2}\int \abs{k}^2 \e^{\i(k,x)} p_t(x)~\mathrm{d}x
\end{align*}
(...)
and conclude by integration by parts 2 times in all coordinates.
In order to conclude the desired equality we use that Fourier transform is an isomorphism which shows that both terms coincide almost everywhere  and as they are continuous they coincide everywhere.
\end{proof}

Main Theorem for today
\begin{thm}
Let $(B_t)$ be a Brownian Motion, $\F_t$ admissible.
Let $f \in C([0,\infty)\times \Rd,\R)\cap C((0,\infty) \times \Rd,\R)$ and assume that there exists $C<\infty$ and a locally bounded function $t \mapsto c(t)$ with
\begin{align*}
\max\{\partial_t f(t,x),\partial_{x_i}^j f(t,x),f(t,x)\colon 1 \leq i \leq d,j=1,2\}\leq c(t)\e^{C\abs{x}}.
\end{align*}
technical assumption, most functions do satisfy this
Define for $t\geq0$
\begin{align*}
M_t^f\coloneqq f(t,B_t)-f(0,B_0)-\int_0^t (Lf)(r,B_r)~\mathrm{d}r
\end{align*}
where $(Lf)(t,x)=\partial_t f(t,x)+\frac{1}{2}\Delta f(t,x)$.
Then $(M_t^f)_{t \geq 0}$ is an $\F_t$ martingale.
It is called the \emph{fundamental martingale} wrt $f$.
\end{thm}
For the proof we note the following lemma, which we have already seen in the discrete case.
\begin{lem}[Doob's maximal inequality]
Let $(M_t)$ be a nonnegative submartingale with continuous paths. Then for all $t\geq 0$ and $p>1$ we have
\begin{align*}
\E{\sup_{s\leq t} M_s^p}\leq \left(\frac{p}{p-1}\right)^p \E{M_t^p}
\end{align*}
\end{lem}
\begin{proof}
exercise, use discrete Martingales and 4.39 from Probability Theory.
\end{proof}

\begin{proof}[Proof of Theorem 2.14]
Fix $\overline{\omega}$ and calculate
\begin{align*}
\E{M_t^f\mid \F_s}(\overline{\omega})&=f(s,B_s(\overline{\omega}))-f(0,B_0(\overline{\omega}))\\
-&\int_0^t (Lf)(r,B_r(\overline{\omega}))~\mathrm{d}r +\E{f(t,B_t)-f(s,B_s)-\int_s^t (Lf)(r,B_r)~\mathrm{d}r\mid \F_s}(\overline{\omega})=(\ast).
\end{align*}
By Proposition 2.23 of PT we know that if $X \amalg Y$ then
\begin{align*}
\E{h(X,Y)\mid \sigma(Y)}(\overline{\omega})=\E{h(X,Y(\overline{\omega})}.
\end{align*}
Applying this with $X=(B_r-B_s)_{r\geq s}$, $Y(s)=(B_r)_{r \leq s}$
%bzgl welches s?!
and
\begin{align*}
h(x,y)=f(t,B_{s,t}-B_s)-f(s,B_s)-\int_s^t (Lf)(r,B_{r,s}-B_s)~\mathrm{d}r
\end{align*}
we obtain from above
\begin{align*}
(\ast)&=M_s^f(\overline{\omega})+\E{f(t,B_{s,t}-B_s(\overline{\omega})}-f(s,B_s(\overline{\omega})-\int_s^t (Lf)(r,B_{r,s}-B_s(\overline{\omega}))~\mathrm{d}r\\
&=M_s^f(\overline{\omega})+\E{\tilde{f}_{\overline{\omega}}(t-s,B_{t-s})-\tilde{f}_{\overline{\omega}}(0,0)-\int_0^{t-s} (L\tilde{f}_{\overline{\omega}}(r,B_{r-s})~\mathrm{d}r}
\end{align*}
with $\tilde{f}_{\overline{\omega}}(u,x)=f(u+s,x-B_s(\overline{\omega}))$.
We used that $L$ commuters with shifts in the parameter space.
The function $\tilde{f}_{\overline{\omega}}$ also fulfills of Thm 2.15, so the theorem will be proved if we can show $\E{M_t^f}=0$ for all allowed $f$, all $t$.
Let $\varepsilon>0$ be arbitrary. Then by Fubini and the assumptions on $f$ we have
\begin{align*}
\E{\int_\varepsilon^t(Lf)(s,B_s)~\mathrm{d}s}&=\int_\varepsilon^t \E{(Lf)(s,B_s)}~\mathrm{d}s\\
&=\int_\varepsilon^t \int p_s(x) (Lf)(s,x)~\mathrm{d}x~\mathrm{d}s\\
&=\int \mathrm{d}x \int_\varepsilon^t p_s(x)(\partial_s f)(s,x)~\mathrm{d}s+\int_\varepsilon^t \mathrm{d}s \int p_s(x)(\frac{1}{2}\Delta f)(s,x)= (...)
\end{align*}
Hence, we get $\E{M_t^f}=\E{f(\varepsilon,B_\varepsilon)-f(0,B_0)}-\E{\int_0^\varepsilon (Lf)(s,B_s)~\mathrm{d}s}$ for all $t$ and $\varepsilon>0$.
%bis hier vl am 5.11
It remains to show $\E{M_t^f}\to 0$ as $\varepsilon \to 0$.
We estimate the last summand using Jensen's formula and obtain
\begin{align*}
\abs{\E{\int_0^\varepsilon (Lf)(s,B_s)~\mathrm{d}s}}& \leq \E{\int_0^\varepsilon \abs{Lf(s,B_s)}~\mathrm{d}s}\\
&\leq \int_0^\varepsilon c(s)\E{\e^{\abs{B_s}}}~\mathrm{d}s \to 0
\end{align*}
as $\varepsilon<0$ as the integrand can be bounded uniformly by $\E{\e^{\abs{B_\varepsilon}}}$ multiplied by some constant, since $\e^{\abs{B_s}}$ is a submartingale.
%why is this the case?
For the other expression, Doobs maximal inequality will be the key tool.
We have
\begin{align*}
\E{\sup_{s\leq \varepsilon}\abs{f(s,B_s)}} &\leq \sup_{s\leq \varepsilon} c(s)\E{\sup_{s\leq \varepsilon} \left( \e^{\frac{1}{2}C \abs{B_s}^2}\right)}\\
&\leq \sup_{s\leq \varepsilon} c(s) \cdot 4 \cdot  \E{\e^{C \abs{B_s}}}.
\end{align*}
%TODO was genau passiert hier
By the continuity of $f$ and $t\mapsto B_t$ we have
\begin{align*}
\lim_{\varepsilon \to 0} f(\varepsilon,B_\varepsilon(\omega))=f(0,0)
\end{align*}
for all $\omega$.
The dominated convergence theorem then yields $\E{f(\varepsilon,B_\varepsilon)} \to f(0,0)$ as $\varepsilon \to 0$.
\end{proof}

\begin{bsp}
\begin{enumerate}[label=\alph*)]
\item Let $f(x,t)=x^2$. Then $Lf=1$ and we have $M_t^f=B_t^2-t$ (see 2.12 a)).
\item Let $f(x,t)=x^3$. Then $Lf=3x$ and $M_t^f=B_t^3-3 \int_0^t B_s~\mathrm{d}s$.
\item Let $f \colon \Rd \to \R$ be harmonic, \ie $\Delta f=0$.
Then $Lf=0$ and hence $(f(B_t))_t$ is a martingale.
For example for $d=2$ and $f(x,y)=\e^x\sin(y)$ then for any two-dimensional Brownian motion also $(\e^{{B_t}^{(1)}}\sin(B_t^{(2)}))_{t\geq 0}$ is a martingale.
\item Let $\mathbf{B_t}=B_t^{(1)}+\i B_t^{(2)}$ be a complex Brownian motion where $B_t^{(j)}$ are independent BMs.
Let $h$ be an entire function with not too much growth at $\infty$
%whatever this should mean...
Then $(h(\mathbf{B_t}))_{t \geq 0}$ is a martingale.
The reason for this is that the real as well as the imaginary part of $h$ are harmonic (due to the Cauchy-Riemann differential equations).
\end{enumerate}
\end{bsp}

\section*{Stopping times}

\begin{defi}
Let $(\F_t)$ be a filtration. A \emph{random time} is a $\R_0^+\cup \{\infty\}$-valued random variable.
A \emph{stopping time} with respect to $(\F_t)$ is a random time $\tau$ with $\{\tau \leq t\} \in \F_t$ for all $t \in \R_0^+$.
\end{defi}

\begin{bsp}
Let $(X_t)$ be an $(\F_t)$-adapted stochastic process with continuous paths and $F\subseteq \Rd$ a closed set. Then the random variable
\begin{align*}
\tau_F\coloneqq \inf\{s\geq 0\colon X_s \in F\}
\end{align*}
is a stopping time called the \emph{hitting time of $F$}.
\end{bsp}
\begin{proof}
Define $d(x,F)=\inf\{\abs{x-y}\colon y \in F\}$ which is continuous in $x$.
%distance as mathoperator?!
By definition we have $\tau_F(\omega)\leq t$ if and only if $\inf\{s\geq 0 \colon X_s(\omega)\in F\} \leq t$. As $F$ is continuous and $(X_t)$ has continuous paths this is equivalent to the existence of $x\in F, s\leq t$ such that $X_s=x$.
Using the continuity of the distance function an equivalent formulation is
\begin{align*}
\inf\{d(X_s(\omega),F),s\in [0,t]\}=0.
\end{align*}
Due to the continuity we can restrict ourselves to $s\in [0,t9\cap \Q$.
Since $\{\omega \colon \inf\{d(X_s(\omega),F),s\in [0,t]\}=0\} \in \F_t$, the claim follows.
%TODO wieso?!
\end{proof}

Obviously, this proof does not work for open sets, so we need a larger filtration.

\begin{defi}
Let $\X$ be a stochastic process. The filtration (check this!)
\begin{align*}
\F_{t+}^\X\coloneqq \bigcap_{u>t} \F_u^\X=\bigcap_{n=1}^\infty \F_{t+\frac{1}{n}}^\X
\end{align*}
is called the \emph{right-continuous completion} of $(\F_t^\X)$.
Any filtration with the property $\F_t=\bigcap_{n\geq 1} \F_{t+\frac{1}{n}}$ is called \emph{right-continuous.}
\end{defi}

Now we can formulate an analogous result for open sets.
\begin{lem}
Let $\X$ be a stochastic process with right-continuous paths and let $U \subseteq \Rd$ be open.
Then
\begin{align*}
\tau_U \coloneqq \inf\{s\geq 0\} \colon X_s \in U\}
\end{align*}
is an $(\F_{t+})$-stopping time.
\end{lem}
\begin{proof}
We have
\begin{align*}
\{\tau_U \leq t\} &= \bigcap_{n\in \N} \{\tau_U<t+\frac{1}{n}\}\\
&=\bigcap_{n\in \N}\bigcup_{j<t+\frac{1}{n}} \{X_j \in U\}\\
&=\bigcap_{n\in \N}\bigcup_{j \in [0,\frac{1}{n})\cap \Q} \{X_j \in U\} \in \F_{t+}. \qedhere
\end{align*}
\end{proof}

\begin{bsp}
Let $(B_t)$ be a two-dimensional Brownian motion and $\tau_b\coloneqq \inf\{t \geq 0 \colon B_t=b\}$ for $b\in \R^2$ be the \emph{first passage time of $b$}.
Then $\Prob(\tau_b<\infty)=1$ for all $b$.
In fact, $t \mapsto B_t(\omega)$ infinitely often for almost all $\omega$. This follows from the next proposition.
%TODO b should from \R^2 as b_t ist 2-dim, or?! or is b_t 1-dimensional which would makes sense with Proposition.
\end{bsp}

\begin{prop}
Let $(B_t)$ be a one-dimensional Brownian motion.
Then
\begin{align*}
\Prob\left(\limsup_{t \to \infty} \frac{1}{\sqrt{t}} B_t=\infty\right)=\Prob\left(\liminf_{t \to \infty} \frac{1}{\sqrt{t}} B_t=\infty\right)=1.
\end{align*}
Moreover, for all $t_0 \in \R_0^+$ we have
\begin{align*}
\Prob\left(\omega \colon B \text{ is not differentiable at } t_0 \right)=1.
\end{align*}
\end{prop}

\begin{proof}
It is clear that
\begin{align*}
\limsup_{t \to \infty} \frac{1}{\sqrt{t}} B_t &\geq \limsup_{n \in \N, n\to \infty} \frac{1}{\sqrt{n}}B_n \\
&= \limsup_{n\to \infty} \frac{1}{\sqrt{n}} \sum_{k=1}^n B_{k,k-1}
\end{align*}
holds true where the $B_{k,k-1}$ are \iid random variables with variance $1$.
By [PT, (2.50a)] we have $\Prob\left(\limsup_{n\to \infty} \frac{1}{\sqrt{n}} B_n=\infty\right)=1$ which implies that the first two statements hold.
For the third, note that by the time shift invariance it is enough to consider $t_0=0$.
By (2.6) we have
\begin{align*}
\Prob\left(\limsup_{t \to \infty} \frac{1}{\sqrt{t}}B_t =\infty\right)&=\Prob\left(\limsup_{t \to \infty} \sqrt{t}B_{\frac{1}{t}}=\infty\right)\\
&=\Prob\left(\limsup_{t \to \infty} \frac{1}{\sqrt{t}} t B_{\frac{1}{t}}=\infty\right)=1.
\end{align*}
The same calculation can be done for $\liminf$.
Hence, $\frac{1}{t}(B_t-B_0)$ does not converge almost surely.
\end{proof}

\begin{bem}
It follows that $\Prob(t\to B_t \text{ is differentiable at any } t \in \Q_0^+)=0$.
With more work, one can show
\begin{align*}
\Prob(\lambda(\{t \in \R_0^+ \colon B \text{ differentiable at } t\})=0)=1.
\end{align*}
With even more work, one can show
\begin{align*}
\Prob(\exists t \geq 0 \colon B \text{ differentiable at } t)=0.
\end{align*}
\end{bem}

Now, we return to considering stopping times.

\begin{prop}
Let $\tau, \sigma$ and $(\tau_n)_{n\in \N}$ with respect to the filtration $(\F_t)$. Then we have
\begin{enumerate}[label=\alph*)]
\item $\{\tau <t\} \in \F$ for all $t$ and if $(\F_t)$ is right-continuous then $\{\rho <t\} \in \F_t$ for all $t$ then $\rho$ is a stopping time.
\item $\tau+ \sigma$, $\tau \wedge \sigma$, $\tau \lor \sigma$ and $\sup_{n\in \N} \tau_n$ are $\F_t$-stopping times.
\item If $(\F_t)$ is right-continuous then $\inf_n \tau_n$, $\liminf_{n \to \infty} \tau_n$ and $\limsup_{n \to \infty} \tau_n$ are stopping times.
\end{enumerate}
\end{prop}
\begin{proof}
This remains as an exercise.
\end{proof}

\begin{lem}[Approximation]
Let $\tau$ be a stopping time. For $n,m \in \N_0$ define $\tau_n(\omega)=(m+1)2^{-n}$ if $m2^{-n}\leq \tau(\omega)<(m+1)2^{-n}$ 
and $\tau_n(\omega)=\infty$ if $\tau(\omega)=\infty$.
Then $\tau_n$ is an $(\F_t)$-stopping time, taking countably many values.
Also $\tau_n(\omega)\searrow \tau(\omega)$ for all $\omega \in \Omega$.
\end{lem}
\begin{proof}
Also this remains as an exercise.
\end{proof}

\begin{defi}
Let $(\F_t)$ be a filtration and $\tau$ be an $(\F_t)$-stopping time.
The $\sigma$-algebra (check this!)
\begin{align*}
\F_t \coloneqq \{A \in \F \colon A \cap \{\tau \leq t\} \F_t \forall t \geq 0\}
\end{align*}
is called the \emph{$\sigma$-algebra of $\tau$'s past}.
\end{defi}

%ab hier vl am 12.11
%letztes vorher: defi 2.27 sigma algebra of the past
\begin{bsp}
Let $C^0(\R_0^+,\Rd)$ and $\F_t=\sigma(\pi_s \colon s\leq t)$ then for $A \in \F=\sigma\left(\cup_{t\geq 0}\F_t\right)$, we have
$A \in \F_t$ if and only if for all $\omega \in \Omega$, the values of $(\pi_s)_{s\leq t}$ determine whether $\omega \in A$.
Moreover, we have $A \in \F_\tau$ if and only if for all $\omega \in \Omega$
such that $\tau(\omega)\leq t$ the values of $(\pi_s)_{s\leq t}$ determine whether $\omega \in A$.
In the case that $\tau(\omega)$ there is no restriction.
Hence, the first condition is the important one.
For example consider closed sets $F,G,H\subseteq \Rd$ and their entry times
$\tau_F,\tau_G,\tau_H$ as in (2.20).
Check whether the set
\begin{align*}
\{\omega \colon t&\mapsto \pi_t(\omega)\text{ first hits } F\text{, then } G\text{, then } H\}
\end{align*}
is in $\F_\sigma$ where $\sigma=\tau_F,\tau_G,\tau_H$.
Do similarly for
\begin{align*}
\{\omega \colon t \mapsto \pi_t(\omega) \text{ has not entered } G \text{ before entering } F\}. 
\end{align*}
\end{bsp}

\begin{prop}
Let $(\F_t)$ be a filtration and $\tau,(\tau_n)$ be $(\F_t)$-stopping times. Then
\begin{enumerate}[label=\alph*)]
\item $\tau \in m \F_\tau$
%TODO own command for measurable?!
\item If $(\F_t)$ is right-continuous, then
\begin{align*}
\F_\tau=\{A \in \F \colon A\cap \{\tau<t\}\in \F_t \forall t\}.
\end{align*}
Note that in the definition we needed to consider $\{\tau\leq t\}$.
\item If $\tau_1(\omega)\leq \tau_2(\omega)$ for all $\omega$ then this ordering also holds for the corresponding $\sigma$-algebras, \ie $\F_{\tau_1}\subseteq \F_{\tau_2}$.
\item We have the following approximation property: If $(\F_\tau)$ is right-continuous and $\tau_n \searrow \tau$ then
\begin{align*}
\F_\tau=\bigcap_{n\in \N} \F_{\tau_n}.
\end{align*}
\item If $(X_t)$ is $(\F_t)$-adapted, $(\F_t)$ right-continuous and $t \mapsto X_t(\omega)$ is right-continuous, then the map
\begin{align*}
\omega \mapsto X_{\tau(\omega)}(\omega) \mathds{1}_{\{\tau_(\omega)<\infty\}},
\end{align*}
\ie the stopped process, is measurable with respect to $\F_\tau$.
%this last property is like the motivation for the definition of sigma-algebra of the past, it tells us the
%TODO google in this context diffusion limited aggregation, maybe include it in the text
\end{enumerate}
\end{prop}
\begin{proof}
a),b),c) are exercises
\begin{itemize}
\item[d)] By c) we have $\F_\tau\subseteq \cap_{n\in \N} \F_{\tau_n}$.
Conversely, for $A \in \F_\tau$ we have
\begin{align*}
A \cap \{\tau<t\}=\bigsqcup_{n\in \N} A \cap \{\tau_n<t\} \in \F_t.
\end{align*}
%wo haben wir right-continuous verwendet?!
\item[e)]
Assume first that $\tau(\Omega)$ is countable.
Then for all $B \in \Borel(\Rd)$ we have
\begin{align*}
\{X_\tau \in B\}\cap \{\tau\leq t\}=\bigsqcup_{k \colon t_k \leq t} \{\tau=t_k,X_{t_k}\in B\} \in \F_t.
\end{align*}
%have converted random stopping time into something deterministic
For general $\tau$ we approximate as in (2.26), \ie
$\tau_n(\omega)\searrow \tau(w)$ for all $\omega \in \Omega$.
By the right-continuity of $t \mapsto X_t(\omega)$, we have
\begin{align*}
X_{\tau_n(\omega)}(\omega)\to X_{\tau(\omega)}(\omega)
\end{align*}
as $n\to \infty$ for all $\omega \in \Omega$.
So, for all $n\geq m$ it holds that
\begin{align*}
X_{\tau_n} \in m \F_{\tau_n}\subseteq m \F_{\tau_m}
\end{align*}
hence $X_\tau \in m \F_{\tau_m}$ which implies $X_\tau \in m \cap_{m\geq 0} \F_{\tau_m}$ which by d) is equal to $\F_\tau$.
\end{itemize}
\end{proof}

Next, we state and prove a very important theorem.
We only formulate it for submartingales but by considering the negative process it holds with reversed inequalities for supermartingales and hence with equality for martingales.
\begin{thm}[Doobs optional stopping theorem]
Let $(X_t)$ be a $(\F_t)$-submartingale, $\sigma,\tau$ be $(\F_t)$-stopping times.
\begin{enumerate}[label=\alph*)]
\item The process $(X_{\tau \wedge k})_{k\in \N}$ is an $(\F_k)_{k\in \N}$ and an $(\F_{\tau \wedge k})$-submartingale.
\item If there exists $T<\infty$ with $\sigma(\omega)\leq \tau(\omega)\leq T$ for all $\omega \in \Omega$ (\ie the stopping times are bounded) then
\begin{align*}
\E{X_\tau\mid \F_\sigma}(\omega)\geq X_{\sigma(\omega)}(\omega)
\end{align*}
holds $\Prob$-almost surely.
\end{enumerate}
\end{thm}
\begin{proof}
We need to transfer known statements from discrete martingales and this remains as an exercise.
\end{proof}
\begin{bem}
The boundedness restriction on the stopping times seems to be very strong.
But, \eg, the condition $\Prob(\tau<\infty)=1$ is not enough, as the following example shows.
Let $(B_t)_{t\geq 0}$ be a one-dimensional Brownian motion and $\tau=\tau_{\{3\}}$. Then by (2.23) $\Prob(\tau_3<\infty)=1$ and $B_{\tau_3(\omega}(\omega)=3$ for all $\omega \in \{\tau_3<\infty\}$.
So, $3=\E{B_{\tau_3}} \not = \E{B_0}=0$.
Then, statement b) of Theorem 2.30 with $\tau=\tau_3$ and $\sigma\equiv 0$ is false.
The statement of (2.30b) does hold under several weaker conditions, see \eg Thm.  A18 in Schilling-Partzsch or Thm 1.93 in [Ligger-Cts time Markov Processes].
We focus on a useful extension that requires $\E{X_t^2}<\infty$ for all $t$.
\end{bem}

\begin{defi}
Let $(X_t)$ be an $\R$-valued martingale with continuous paths and $\E{X_t^2}<\infty$ for all $t$.
A stochastic process $(A_t)_{t \geq 0}$ is called \emph{quadratic variation process}(qvp) of $(X_t)$ if
\begin{enumerate}[label=(\roman*)]
\item $A_0=0$,
\item $t \mapsto A_t(\omega)$ is increasing almost surely,
\item $(X_t^2-A_t)_{t \geq 0}$ is a martingale.
\end{enumerate}
\end{defi}
\begin{bem}
\begin{enumerate}[label=\alph*)]
\item Proposition 2.12 a) shows that $A_t(\omega)\coloneqq t$ defines a qvp for a one-dimensional Brownian motion.
\item For discrete martingales, Thm 4.49 of [PT] shows that $(A_n)_{n\in \N}$ is unique and is given by some nice formula which can be found there
%TODO maybe include it here, too, lazy to include it in the lecture.
\item For continuous martingales, there also exists a unique stochastic process fulfilling the preceding definition.
It is given by
\begin{align*}
A_t= \lim_{\abs{\pi}\to \infty} \sum_{t_i \in \pi}(X_{t_i}-X_{t_{i-1}})^2
\end{align*}
where $\pi$ denotes some partition. The limit is independent of the approximating sequence of partitions and exists in $\mathit{L}^2$.
The proof is long and can be found in [Ligget, Thm 5.3].
\end{enumerate}
\end{bem}

\begin{thm}
Let $(X_t)$ be a continuous martingale, $(A_t)$ its qvp and $\sigma,\tau$ be stopping times with $\sigma\leq \tau$.
If $\Prob(\tau<\infty)=1$ and $\E{A_\tau}<\infty$, then
\begin{enumerate}[label=\alph*)]
\item $\E{X_\tau \mid \F_\sigma}=X_\sigma$ almost surely,
\item $\E{X_\tau^2-A_\tau \mid \F_\sigma}=\X_\sigma^2-A_\sigma$ almost surely.
\end{enumerate}
In particular $\E{X_\tau}=\E{X_0}$ and $\E{X_\tau^2}=\E{X_0^2}-\E{A_\tau}$.
\end{thm}
\begin{proof}
Let $n\geq k$. Since $\F_{\tau \wedge n}\supseteq \F_{\tau \wedge k} \supseteq \F_{\sigma \wedge k}$ and since $\tau \wedge n \leq n <\infty$, for any martingale $(M_t)$ we have by the tower property
\begin{align*}
\E{M_{\tau \wedge n} \mid \F_{\sigma \wedge k}}&=\E{\E{M_{\tau\wedge n} \mid \F_{\tau \wedge k}} \mid \F_{\sigma \wedge k}}\\
&=\E{M_{\tau \wedge k} \mid \F_{\sigma\wedge k}}=M_{\sigma \wedge k},
\end{align*}
where we have used 2.30 a) and  b) in the last two steps.
Applying this for $\sigma = \tau, M_t=X_t^2-A_t$ we obtain by the calculations above
\begin{align*}
\E{\left(X_{\tau \wedge n}-X_{\tau \wedge k}\right)^2}&=\E{X_{\tau\wedge n}^2 -X_{\tau \wedge k}^2}\\
&=\E{A_{\tau \wedge n}- A_{\tau \wedge k}}\\
&=\E{(A_{\tau \wedge n}- A_{\tau\wedge k} \mathds{1}_{\{\tau \geq k\}}} \overset{n,k \to \infty}{\longrightarrow} 0.
\end{align*}
This shows that $(X_{\tau \wedge n})_{n \in \N}$ is a Cauchy sequence in $\mathit{L}^2$.
By the continuity of $X$ we have $X_{\tau \wedge n} \to \X_{\tau}$ almost surely and so $X_{\tau \wedge n} \to \X_\tau$ in $\mathit{L}^2$ and in particular
\begin{align*}
X_{\tau \wedge n}^2- A_{\tau \wedge n} \to X_\tau^2 -A_\tau
\end{align*}
in $\mathit{L}^1$ as $n \to \infty$.
So, $M_{\tau \wedge n} \to M_tau$ in $\mathit{L}^1$ both holds for $M=X$ and $M=X^2-A$.
Thus, for all $k$ and all $n\geq k$ we have using the tower property
\begin{align*}
\E{M_\tau\mid \F_{\sigma \wedge k}}&=\E{\E{M_\tau \mid \F_{\tau \wedge n}} \mid \F_{\sigma\wedge k}}\\
&=M_{\sigma\wedge k}.
\end{align*}
By Thm 4.68 of [PT] we have
\begin{align*}
\E{M_\tau\mid \F_{\sigma \wedge k}} \to \E{M_\tau \mid \F_\sigma}
\end{align*}
almost surely and in $\mathit{L}^1$ as $k\to \infty$ and we already know that
$M_{\sigma \wedge k} \to M_\sigma$ in $\mathit{L}^1$.
This shows a),b). The final statement follow by taking $\sigma \equiv 0$ and taking expectation. 
\end{proof}

\begin{cor}[Wald's identities]
Let $(B_t)$ be a one-dimensional Brownian Motion and $\tau$ a stopping time.
if $\E{\tau}<\infty$, then
\begin{align*}
\E{B_\tau}=0, \E{B_\tau^2}=\E{\tau}
\end{align*}
\end{cor}
\begin{proof}
$B_t^2-t$ martingale, use (2.34)
note $A_{\tau(\omega)}=\tau(\omega) \Rightarrow \E{A_\tau}=\E{\tau}$.
\end{proof}